%
% v. 0.2 - 11/08/2009 - Jarek - extended introductions and bibliography
\documentclass[final,5p,times,twocolumn]{elsarticle}

%% Use the option review to obtain double line spacing
%% \documentclass[authoryear,preprint,review,12pt]{elsarticle}

%% Use the options 1p,twocolumn; 3p; 3p,twocolumn; 5p; or 5p,twocolumn
%% for a journal layout:
%% \documentclass[final,1p,times]{elsarticle}
%% \documentclass[final,1p,times,twocolumn]{elsarticle}
%% \documentclass[final,3p,times]{elsarticle}
%% \documentclass[final,3p,times,twocolumn]{elsarticle}
%% \documentclass[final,5p,times]{elsarticle}
%% \documentclass[final,5p,times,twocolumn]{elsarticle}

%% if you use PostScript figures in your article
%% use the graphics package for simple commands
%% \usepackage{graphics}
%% or use the graphicx package for more complicated commands
%% \usepackage{graphicx}
%% or use the epsfig package if you prefer to use the old commands
%% \usepackage{epsfig}

%% The amssymb package provides various useful mathematical symbols
\usepackage{amssymb}
%% The amsthm package provides extended theorem environments
%% \usepackage{amsthm}

%% The lineno packages adds line numbers. Start line numbering with
%% \begin{linenumbers}, end it with \end{linenumbers}. Or switch it on
%% for the whole article with \linenumbers.
%% \usepackage{lineno}

\newcommand{\Mathematica}{\emph{Mathematica}}
\newcommand{\wek}{\mathbf{vec}}
\newcommand{\res}{\mathbf{res}}
\newcommand{\1}{{\rm 1\hspace{-0.9mm}l}}
\newcommand{\Id}{\1}
\newcommand{\SWAP}{\ensuremath{\mathrm{SWAP}}}
\newcommand{\tr}{\mathrm{tr}}
\newcommand{\M}{\ensuremath{\mathbb{M}}}
\newcommand{\qi}{QI}

\usepackage{xcolor}
\newcommand{\todo}[1]{\textcolor{red}{\bf TODO: #1}}

\journal{Computer Physics Communications}

\begin{document}

\begin{frontmatter}

%% Title, authors and addresses

%% use the tnoteref command within \title for footnotes;
%% use the tnotetext command for theassociated footnote;
%% use the fnref command within \author or \address for footnotes;
%% use the fntext command for theassociated footnote;
%% use the corref command within \author for corresponding author footnotes;
%% use the cortext command for theassociated footnote;
%% use the ead command for the email address,
%% and the form \ead[url] for the home page:
%% \title{Title\tnoteref{label1}}
%% \tnotetext[label1]{}
%% \author{Name\corref{cor1}\fnref{label2}}
%% \ead{email address}
%% \ead[url]{home page}
%% \fntext[label2]{}
%% \cortext[cor1]{}
%% \address{Address\fnref{label3}}
%% \fntext[label3]{}

\title{Analysis of quantum states and operations with \qi\ package\\[6pt] \small{v. 0.2 (11 August 2009)}}

%% use optional labels to link authors explicitly to addresses:
%% \author[label1,label2]{}
%% \address[label1]{}
%% \address[label2]{}

\author{P.~Gawron}
\author{J.~A.~Miszczak}
\author{Z.~Pucha{\l}a\corref{cor1}}
\ead{z.puchala@iitis.pl}
\cortext[cor1]{Corresponding Author}
\address{Institute of Theoretical and Applied Informatics, Polish Academy of 
Sciences, Ba{\l}tycka 5, 44-100 Gliwice, Poland}

\begin{abstract}
We present a package of functions for \Mathematica\ computer algebra system, 
which implements number of functions used in the analysis of quantum states.

In contrast to many available packages for symbolic and numerical simulation of
quantum computation presented package is focused on geometrical aspects of 
quantum information theory.

\end{abstract}

\begin{keyword}
%% keywords here, in the form: keyword \sep keyword
quantum states \sep quantum operations \sep partial operations
%% PACS codes here, in the form: \PACS code \sep code
%quantum information, symbolic computation (computer algebra) 
\PACS 03.67.-a \sep 02.70.Wz

%% MSC codes here, in the form: \MSC code \sep code
%% or \MSC[2008] code \sep code (2000 is the default)

\end{keyword}

\end{frontmatter}

%% \linenumbers

%% main text
%%%%%%%%%%%%%%%%%%%%%%%%%%%%%%%%%%%%%%%%%%%%%%%%%%%%%%%%%%%%%%%%%%%%%%%%%%%%%%%%
\section{Introduction}\label{sec:intro}
%%%%%%%%%%%%%%%%%%%%%%%%%%%%%%%%%%%%%%%%%%%%%%%%%%%%%%%%%%%%%%%%%%%%%%%%%%%%%%%%
Quantum information theory aims to harness behavior of quantum mechanical
objects to store, transfer and process information~\cite{hayashi}. This behavior
is in many cases very different from the one we observe in classical world.
Quantum algorithms and protocols take advantage of superposition of states and
require the presence of entangled states. Both phenomena arise from the rich
structure of the space of quantum states. Hence, to explore the capabilities of
quantum information processing, one needs to fully understand this
space~\cite{BZ06}. 

On the other hand quantum mechanics provides us also with much larger space of
allowed operations which can be used to manipulate quantum states~\cite{BZ06}.
Recent results concerning additivity problems~\cite{hastings09superadditivity}
show that we are far from full understanding the nature of quantum channels.
Exploring the space of quantum operations is fascinating, but cumbersome
task~\cite{ruskai02analysis}

%One of the motivations behind quantum information theory was an observation that
%the simulation of quantum systems on classical machine requires exponential 
%resources.

We present a~package of functions developed for \Mathematica\ 7.0 which aims to
simplify the analysis of quantum states and quantum operations. The package was 
developed in simplicity in mind and thus it uses only basic data structures
available in \Mathematica. This allows to relatively easily port implemented
functions to other scientific software systems.

This paper is organized as follows. In Section~\ref{sec:qi-intro} we review
basic notion of quantum density matrices and quantum channels representing
allowed physical transformations of density matrices. Section~\ref{sec:over}
presents an overview of functionality provided by \qi\ package and describes
some operations used as building blocks in specialized functions.
Sections~\ref{sec:states} and \ref{sec:channels} provide detailed description of
functions implemented in~\qi. Finally Section~\ref{sec:comclude}.

%% The Appendices part is started with the command \appendix;
%% appendix sections are then done as normal sections
%% \appendix

%%%%%%%%%%%%%%%%%%%%%%%%%%%%%%%%%%%%%%%%%%%%%%%%%%%%%%%%%%%%%%%%%%%%%%%%%%%%%%%%
\section{Quantum states and operations}\label{sec:qi-intro}
%%%%%%%%%%%%%%%%%%%%%%%%%%%%%%%%%%%%%%%%%%%%%%%%%%%%%%%%%%%%%%%%%%%%%%%%%%%%%%%%



%%%%%%%%%%%%%%%%%%%%%%%%%%%%%%%%%%%%%%%%%%%%%%%%%%%%%%%%%%%%%%%%%%%%%%%%%%%%%%%%
\section{Overview of \qi}\label{sec:over}
%%%%%%%%%%%%%%%%%%%%%%%%%%%%%%%%%%%%%%%%%%%%%%%%%%%%%%%%%%%%%%%%%%%%%%%%%%%%%%%%
\todo{troche o zalozeniach projektowych czyli dlaczego \qi\ jest inne niz pozostale pakiety}
\newcounter{principle}
\begin{list}{(P\Roman{principle})}{\usecounter{principle}}
\item user know what she/he is doing -- function do not expect to get
positive/unitary/\ldots\ matrix 
\item only basic \Mathematica\ structures are
used -- functions operator on plain \Mathematica\ lists and return lists 
\item as simple as possible -- functions have minimum reasonable number of
arguments
\item only basic version are provided -- functions perform only simple
operations, for example in order to get partial trace on few subsystems user
needs to perform permutation and then apply partial trace 
\item some data are
used more often then other -- \qi\ predefines some commonly used matrices, for
example matrix for \SWAP\ operation for some small dimensions.
\end{list}


%%%%%%%%%%%%%%%%%%%%%%%%%%%%%%%%%%%%%%%%%%%%%%%%%%%%%%%%%%%%%%%%%%%%%%%%%%%%%%%%
\section{Quantum states in \qi}\label{sec:states}
%%%%%%%%%%%%%%%%%%%%%%%%%%%%%%%%%%%%%%%%%%%%%%%%%%%%%%%%%%%%%%%%%%%%%%%%%%%%%%%%

%%%%%%%%%%%%%%%%%%%%%%%%%%%%%%%%%%%%%%%%%%%%%%%%%%%%%%%%%%%%%%%%%%%%%%%%%%%%%%%%
\subsection{One-qubit states}
%%%%%%%%%%%%%%%%%%%%%%%%%%%%%%%%%%%%%%%%%%%%%%%%%%%%%%%%%%%%%%%%%%%%%%%%%%%%%%%%

%%%%%%%%%%%%%%%%%%%%%%%%%%%%%%%%%%%%%%%%%%%%%%%%%%%%%%%%%%%%%%%%%%%%%%%%%%%%%%%%
\subsection{Generalizations}
%%%%%%%%%%%%%%%%%%%%%%%%%%%%%%%%%%%%%%%%%%%%%%%%%%%%%%%%%%%%%%%%%%%%%%%%%%%%%%%%
\cite{tilma02generalized}


%%%%%%%%%%%%%%%%%%%%%%%%%%%%%%%%%%%%%%%%%%%%%%%%%%%%%%%%%%%%%%%%%%%%%%%%%%%%%%%%
\section{Quantum operations in \qi}\label{sec:channels}
%%%%%%%%%%%%%%%%%%%%%%%%%%%%%%%%%%%%%%%%%%%%%%%%%%%%%%%%%%%%%%%%%%%%%%%%%%%%%%%%

%%%%%%%%%%%%%%%%%%%%%%%%%%%%%%%%%%%%%%%%%%%%%%%%%%%%%%%%%%%%%%%%%%%%%%%%%%%%%%%%
\section{Representation of quantum operations}
%%%%%%%%%%%%%%%%%%%%%%%%%%%%%%%%%%%%%%%%%%%%%%%%%%%%%%%%%%%%%%%%%%%%%%%%%%%%%%%%


%%%%%%%%%%%%%%%%%%%%%%%%%%%%%%%%%%%%%%%%%%%%%%%%%%%%%%%%%%%%%%%%%%%%%%%%%%%%%%%%
\subsection{Partial operations}
%%%%%%%%%%%%%%%%%%%%%%%%%%%%%%%%%%%%%%%%%%%%%%%%%%%%%%%%%%%%%%%%%%%%%%%%%%%%%%%%
\qi\ implements partial operations using general method for constructing channels
acting on subsystems. This method is based on formula
\begin{equation}\label{eqn:def-tensor}
(\Phi\otimes\Id)(\rho) = 
\left(\res^{-1}\left(M_\Phi\res\left(\rho^R\right)\right)\right)^R
\end{equation}
where ${}^R$ denotes the reshuffling operation and $M_\Phi$ denotes the matrix
of the linear map $\Phi$
\begin{equation}
M_\Phi = \tr \epsilon_i \Phi(\epsilon_j),
\end{equation}
where $\{\epsilon_i\}_i=1,\ldots,n^2$ is a canonical basis in $\M_n$. In the 
Eq.~\ref{eqn:def-tensor} operation $\Phi$ is applied to the first subsystem 
only.

%%%%%%%%%%%%%%%%%%%%%%%%%%%%%%%%%%%%%%%%%%%%%%%%%%%%%%%%%%%%%%%%%%%%%%%%%%%%%%%%
\subsubsection{Partial transposition}
%%%%%%%%%%%%%%%%%%%%%%%%%%%%%%%%%%%%%%%%%%%%%%%%%%%%%%%%%%%%%%%%%%%%%%%%%%%%%%%%
In the case of transposition $M_\Phi$ is equivalent to \SWAP. For example in the
case of $4$-dimensional density matrix
\begin{equation}
X=\left(
\begin{array}{cccc}
 \alpha_{1,1} & \alpha_{1,2} & \alpha_{1,3} & \alpha_{1,4} \\
 \alpha_{2,1} & \alpha_{2,2} & \alpha_{2,3} & \alpha_{2,4} \\
 \alpha_{3,1} & \alpha_{3,2} & \alpha_{3,3} & \alpha_{3,4} \\
 \alpha_{4,1} & \alpha_{4,2} & \alpha_{4,3} & \alpha_{4,4}
\end{array}
\right)
\end{equation}
partial transposition with respect to second subsystems is obtained using
\verb+PartialTransposeB[X, 2, 2]+ and it gives
\begin{equation}
X^{T_B}=\left(
\begin{array}{cccc}
 \alpha_{1,1} & \alpha_{2,1} & \alpha_{1,3} & \alpha_{2,3} \\
 \alpha_{1,2} & \alpha_{2,2} & \alpha_{1,4} & \alpha_{2,4} \\
 \alpha_{3,1} & \alpha_{4,1} & \alpha_{3,3} & \alpha_{4,3} \\
 \alpha_{3,2} & \alpha_{4,2} & \alpha_{3,4} & \alpha_{4,4}
\end{array}
\right).
\end{equation}

\todo{dodac druga wersje}
%%%%%%%%%%%%%%%%%%%%%%%%%%%%%%%%%%%%%%%%%%%%%%%%%%%%%%%%%%%%%%%%%%%%%%%%%%%%%%%%
\subsubsection{Partial trace}
%%%%%%%%%%%%%%%%%%%%%%%%%%%%%%%%%%%%%%%%%%%%%%%%%%%%%%%%%%%%%%%%%%%%%%%%%%%%%%%%
Operation of tracing out a subsystem can be achieved in a similar manner. We 
define tracing map as
\begin{equation}
\Phi_\mathrm{tr}(\rho) = \tr \rho \Id,
\end{equation}
which for gives
\begin{equation}
D_{\Phi_\mathrm{tr}} =
\left(
\begin{array}{cccc}
 1 & 0 & 0 & 1 \\
 0 & 0 & 0 & 0 \\
 0 & 0 & 0 & 0 \\
 1 & 0 & 0 & 1
\end{array}
\right)
\end{equation}
for one qubit. Note that this matrix after reshuffling is equal to $\Id$ and 
thus $\Phi_\tr$ is CP-TP map.

\todo{dodac druga wersje}

%%%%%%%%%%%%%%%%%%%%%%%%%%%%%%%%%%%%%%%%%%%%%%%%%%%%%%%%%%%%%%%%%%%%%%%%%%%%%%%%
\section{Concluding remarks}\label{sec:comclude}
%%%%%%%%%%%%%%%%%%%%%%%%%%%%%%%%%%%%%%%%%%%%%%%%%%%%%%%%%%%%%%%%%%%%%%%%%%%%%%%%
Package \qi\ provide set of functions which aims to simplify the task of
exploring the space of quantum states and understanding quantum operations.
  
  
\bibliography{qi}
\bibliographystyle{elsarticle-num}

\end{document}
\endinput
%%
%% End of file `elsarticle-template-num.tex'.
