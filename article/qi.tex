% v. 0.2 - 11/08/2009 - Jarek - extended introductions and bibliography
% v. 0.21 - 12/08/2009 - Jarek - preliminary sections
% v. 0.22 - 13/08/2009 - Jarek - added in intro and sec. 2
% v. 0.3 - 17/11/2009 - Jarek - added an appendix with a list of functions
% v. 0.31 - 18/11/2009 - Jarek - sections on states and operations
% v. 0.32 - 19/11/2009 - Jarek - improved abstact, applied CPC template 
% v. 0.33 - 24/09/2010 - Jarek - update basic info, sync with 0.3.18 version 
\documentclass[a4paper,11pt]{elsart}

\usepackage{hyperref}

%% This list environment is used for the references in the
%% Program Summary
%%
%\newcounter{bla}
%\newenvironment{refnummer}{%
%\list{[\arabic{bla}]}%
%{\usecounter{bla}%
% \setlength{\itemindent}{0pt}%
% \setlength{\topsep}{0pt}%
% \setlength{\itemsep}{0pt}%
% \setlength{\labelsep}{2pt}%
% \setlength{\listparindent}{0pt}%
% \settowidth{\labelwidth}{[9]}%
% \setlength{\leftmargin}{\labelwidth}%
% \addtolength{\leftmargin}{\labelsep}%
% \setlength{\rightmargin}{0pt}}}
% {\endlist}

%%%%%%%%%%%%%%%%%%%%%%%%%%%%%%%%%%%%%%%%%%%%%%%%%%%%%%%%%%%%%%%%%%%%%%%%%%%%%%%%
\usepackage{amsmath,amssymb}
\usepackage{dsfont}
\usepackage{listings}
\newcommand{\ket}[1]{\ensuremath{|#1\rangle}}
\newcommand{\bra}[1]{\ensuremath{\langle#1|}}
\newcommand{\Mathematica}{\emph{Mathematica}}
\newcommand{\wek}{\mathbf{vec}}
\newcommand{\res}{\mathbf{res}}
\newcommand{\1}{{\rm 1\hspace{-0.9mm}l}}
\newcommand{\Id}{\1}
\newcommand{\SWAP}{\ensuremath{\mathrm{SWAP}}}
\newcommand{\tr}{\mathrm{tr}}
\newcommand{\M}{\ensuremath{\mathbb{M}}}
\newcommand{\qi}{QI}
\newcommand{\code}[1]{\texttt{\small #1}}
\newcommand{\HS}[1]{\ensuremath{\mathcal{#1}}} % Hilbert space
\newcommand{\Cplx}{\ensuremath{\mathbb{C}}}
\newcommand{\eg}{\emph{eg.}}
\newcommand{\ie}{\emph{ie.}}
%\newcommand{\etal}{\emph{et al.}}
\newcommand{\Prob}[1]{\ensuremath{\mathrm{P}(#1)}}
\newcommand{\Observ}[1]{\ensuremath{#1}}
\newcommand{\Spectrum}[1]{\ensuremath{\sigma(#1)}}
\newcommand{\Spec}[1]{\Spectrum{#1}}
\newcommand{\ketbra}[2]{\ensuremath{\ket{#1}\bra{#2}}}
\newcommand{\proj}[1]{\ensuremath{\ketbra{#1}{#1}}}
\newcommand{\Proj}[1]{\proj{#1}}
\newcommand{\iner}[2]{\braket{#1}{#2}}
\newcommand{\Iner}[2]{\iner{#1}{#2}}
\newcommand{\braket}[2]{\ensuremath{\langle#1|#2\rangle}}
\newcommand{\Tr}[2][]{\ensuremath{\tr_{#1}{#2}}}
\newcommand{\id}{\mathds{1}}
\newcommand{\Space}[1]{\mathcal{#1}}
\newcommand{\SetOfStates}[1]{\ensuremath{\mathcal{S}(#1)}}
\newcommand{\States}[1]{\SetOfStates{#1}}
\newcommand{\Real}{\ensuremath{\mathds{R}}}
\newcommand{\N}{\ensuremath{\mathds{N}}}
\newcommand{\Z}{\ensuremath{\mathds{Z}}}
\newcommand{\set}[2]{\ensuremath{\left\{#1|#2\right\}}}
\newcommand{\re}[1]{\ensuremath{\mathrm{Re}\left(#1\right)}}
\newcommand{\im}[1]{\ensuremath{\mathrm{Im}\left(#1\right)}}
\newcommand{\Group}[2]{\ensuremath{(#1,#2)}}
\newcommand{\ord}[2]{\mathrm{ord}_#2(#1)}
\newcommand{\Var}[1]{\ensuremath{\mathrm{Var}(#1)}}
\newcommand{\halmos}{\newline\vspace{3mm}\hfill $\Box$}
\newcommand{\proof}{\noindent {\it Proof.\ }}
\newtheorem{theorem}{Theorem}
\newcommand{\stackidx}[4]{
  \substack{
  #1  #2 \\
  #3  #4}
}


\usepackage{xcolor}
\newcommand{\todo}[1]{\textcolor{red}{\bf TODO: #1}}
%%%%%%%%%%%%%%%%%%%%%%%%%%%%%%%%%%%%%%%%%%%%%%%%%%%%%%%%%%%%%%%%%%%%%%%%%%%%%%%%

\journal{Computer Physics Communications}

\begin{document}

\lstset{language=Mathematica,frame=lines}

\begin{frontmatter}

%% Title, authors and addresses

%% use the tnoteref command within \title for footnotes;
%% use the tnotetext command for theassociated footnote;
%% use the fnref command within \author or \address for footnotes;
%% use the fntext command for theassociated footnote;
%% use the corref command within \author for corresponding author footnotes;
%% use the cortext command for theassociated footnote;
%% use the ead command for the email address,
%% and the form \ead[url] for the home page:
%% \title{Title\tnoteref{label1}}
%% \tnotetext[label1]{}
%% \author{Name\corref{cor1}\fnref{label2}}
%% \ead{email address}
%% \ead[url]{home page}
%% \fntext[label2]{}
%% \cortext[cor1]{}
%% \address{Address\fnref{label3}}
%% \fntext[label3]{}

\title{\qi: A package for the analysis of quantum states and operations in
\Mathematica}

\date{15/09/2011 (v. 0.34)}

\author{J.~A.~Miszczak\thanksref{author}},
\ead{miszczak@iitis.pl}
\author{P.~Gawron},
\ead{gawron@iitis.pl}
\author{Z.~Pucha{\l}a}
\ead{z.puchala@iitis.pl}

\thanks[author]{Corresponding author}

\address{Institute of Theoretical and Applied Informatics, Polish Academy of 
Sciences, Ba{\l}tycka 5, 44-100 Gliwice, Poland}

\begin{abstract}
QI is a package of functions for the \Mathematica\ computer algebra system,
which provides a framework for the analysis of quantum states and quantum
operations. In contrast to many available packages for symbolic and numerical
simulation of quantum computation presented package is focused on geometrical
aspects of quantum information theory. In particular \qi provides
parametrization of quantum states and selected families of quantum operation,
function for constructing composite quantum operations and methods for the
generation and analysis of random quantum operations. Also basic structures are
provided including the construction of ket vectors, basic unitary gates, random
states and unitaries, distance measures between quantum states and the selected
methods for the analysis of separability in quantum systems.

\begin{flushleft}
  %Insert your suggested PACS number here
%quantum information, symbolic computation (computer algebra) 
PACS: 03.67.-a; 02.70.Wz.
\end{flushleft}

\begin{keyword}
  % Please give some freely chosen keywords that we can use in a
  % cumulative keyword index.
quantum states; quantum channels; partial operations.
\end{keyword}
\end{abstract}

\end{frontmatter}
%%%%%%%%%%%%%%%%%%%%%%%%%%%%%%%%%%%%%%%%%%%%%%%%%%%%%%%%%%%%%%%%%%%%%%%%%%%%%%%%
{\bf PROGRAM SUMMARY}
  %Delete as appropriate.

\begin{small}
\noindent
{\em Manuscript Title:} \qi: A package for the analysis of quantum states and operations in
\Mathematica\\
{\em Authors:} J.A.~Miszczak, P.~Gawron, Z.~Pucha{\l}a \\
{\em Program Title:} QI \\
{\em Journal Reference:}                                      \\
  %Leave blank, supplied by Elsevier.
{\em Catalogue identifier:}                                   \\
  %Leave blank, supplied by Elsevier.
{\em Licensing provisions:} GPLv3 \\
{\em Programming language:} Mathematica 7\\
{\em Computer:} Any computer supporting Mathematica 7\\
  %Computer(s) for which program has been designed.
{\em Operating system:} Any operating system capable of running Mathematica 7 or higher, \eg\ GNU/Linux, MacOS X, FreeBSD, Microsoft Windows XP\\
  %Operating system(s) for which program has been designed.
{\em RAM:} bytes                                              \\
  %RAM in bytes required to execute program with typical data.
{\em Number of processors used:}                              \\
  %If more than one processor.
{\em Supplementary material:}                                 \\
  % Fill in if necessary, otherwise leave out.
{\em Keywords:} quantum states, quantum operations, partial operations  \\
  % Please give some freely chosen keywords that we can use in a
  % cumulative keyword index.
{\em PACS:} 03.67.-a, 02.70.Wz.\\
  % see http://www.aip.org/pacs/pacs.html
{\em Classification:} 4.15 \\
  %Classify using CPC Program Library Subject Index, see (
  % http://cpc.cs.qub.ac.uk/subjectIndex/SUBJECT_index.html)
  %e.g. 4.4 Feynman diagrams, 5 Computer Algebra.
{\em External routines/libraries:}                                      \\
  % Fill in if necessary, otherwise leave out.
{\em Subprograms used:}                                       \\
  %Fill in if necessary, otherwise leave out.

{\em Nature of problem:}\\
  %Describe the nature of the problem here.
  Construction of composed quantum operations, analysis of quantum states and
  operations.
   \\
{\em Solution method:}\\
  %Describe the method solution here.
  A package of functions for \Mathematica\ computer algebra system.
   \\
{\em Restrictions:}\\
  %Describe any restrictions on the complexity of the problem here.
  Running time of the presented procedures grows rapidliy with the dimensionalty
  of the problem.
   \\
%{\em Unusual features:}\\
%  %Describe any unusual features of the program/problem here.
%   \\
%{\em Additional comments:}\\
%  %Provide any additional comments here.
%   \\
{\em Running time:}\\
  %Give an indication of the typical running time here.
   \\

\end{small}

%\newpage

% In program descriptions the main text of the paper is listed under
% the heading LONG WRITE-UP.

%\hspace{1pc}
%{\bf LONG WRITE-UP}
%%%%%%%%%%%%%%%%%%%%%%%%%%%%%%%%%%%%%%%%%%%%%%%%%%%%%%%%%%%%%%%%%%%%%%%%%%%%%%%%


\tableofcontents

%% \linenumbers

%% main text
%%%%%%%%%%%%%%%%%%%%%%%%%%%%%%%%%%%%%%%%%%%%%%%%%%%%%%%%%%%%%%%%%%%%%%%%%%%%%%%%
\section{Introduction}\label{sec:intro}
%%%%%%%%%%%%%%%%%%%%%%%%%%%%%%%%%%%%%%%%%%%%%%%%%%%%%%%%%%%%%%%%%%%%%%%%%%%%%%%%
Quantum information theory aims to harness the behavior of quantum mechanical
objects to store, transfer and process information~\cite{hayashi}. This behavior
is in many cases very different from the one we observe in classical world.
Quantum algorithms and protocols take advantage of superposition of states and
require the presence of entangled states. Both phenomena arise from the rich
structure of the space of quantum states. Hence, to explore the capabilities of
quantum information processing, one needs to fully understand this
space~\cite{BZ06}. 

Quantum mechanics provides us also with much larger than in classical case space
of allowed operations which can be used to manipulate quantum
states~\cite{hayashi,BZ06}. Recent results concerning additivity
problems~\cite{hastings09superadditivity} show that we are far from full
understanding the nature of quantum channels. Exploring the space of quantum
operations is fascinating, but cumbersome task.

We present a~package of functions developed for \Mathematica\ computing system
which aims to simplify the analysis of quantum states and quantum operations.
The package was developed in simplicity in mind and thus it uses only basic data
structures available in \Mathematica. This allows to relatively easily port
implemented functions to other scientific software systems. Also, in contrast to
most quantum computing packages
available~\cite{qdensity,qucalc,quantum2,qcwave}, \qi\ is not aimed to provide
tool for simulating quantum algorithms and protocols. We rather focus on the
analysis of quantum states used in those protocols and quantum channels, which
are used to describe allowed physical operations. The main goal of presented
package is to provide basic mathematical tools useful for studding geometrical
properties of quantum stats and quantum channels.




%%%%%%%%%%%%%%%%%%%%%%%%%%%%%%%%%%%%%%%%%%%%%%%%%%%%%%%%%%%%%%%%%%%%%%%%%%%%%%%%
\subsection{Design principles}
%%%%%%%%%%%%%%%%%%%%%%%%%%%%%%%%%%%%%%%%%%%%%%%%%%%%%%%%%%%%%%%%%%%%%%%%%%%%%%%%

\newcounter{principle}
\begin{list}{\textbf{P\arabic{principle}}}{\usecounter{principle}}
\item Functions are as simple as possible -- functions have minimum reasonable
number of arguments and implement small pieces of functionality. For example in
order to get partial trace on few subsystems user needs to perform permutation
and then apply partial trace 
\item User knows what she is doing -- functions implemented in \qi\ do not
validate the input and user is expected to provide reasonable data.
\item Only basic \Mathematica\ structures are used -- functions operator on
plain \Mathematica\ lists and return lists.
\item Some data are used more often then other -- \qi\ predefines some commonly
used matrices, for example matrices for \SWAP\ operation and Pauli matrices for
small dimensions.
\end{list}

%%%%%%%%%%%%%%%%%%%%%%%%%%%%%%%%%%%%%%%%%%%%%%%%%%%%%%%%%%%%%%%%%%%%%%%%%%%%%%%
%%
\section{Basic functions}\label{sec:basic}
%%%%%%%%%%%%%%%%%%%%%%%%%%%%%%%%%%%%%%%%%%%%%%%%%%%%%%%%%%%%%%%%%%%%%%%%%%%%%%%
\qi{} provides functions that facilitate working with state vectors and 
density matrices. In this section we will present the basic notation of 
quantum mechanics and quantum information theory together with the way of 
implementing it in \qi.

\subsection{Dirac notation}
The pure state of a~quantum system is described by complex normed vector. It 
is usually denoted by the symbol $\ket{\psi}$. The dual vector to $\ket{\psi}$
is denoted by $\bra{\psi}$. The scalar product of vectors $\ket{\psi}$, 
$\ket{\phi}$ is denoted by $\braket{\phi}{\psi}$. The outer product 
of these vectors is denoted as $\ketbra{\phi}{\psi}.$ 
Vectors are labelled in the natural way: 
$\ket{0}:=\left[1,0,\ldots,0\right]^T, 
\ket{1}:=\left[0,1,\ldots,0\right]^T,\ket{n-1}:=\left[0,0,\ldots,1\right]^T$.
Notation like $\ket{\phi\psi}$ denotes the tensor product of vectors and is
equivalent to $\ket{\phi}\otimes\ket{\psi}$.

\qi{} provides the functions \lstinline!Ket[n,d]! and
\lstinline!KetFromDigits[\{$n_0,n_1,\ldots,n_k$\}, d]! that 
create unit vectors. First one returns $\ket{n}\in\Cplx^d$ second one return 
$\ket{n_0,n_1\ldots,n_k}\in\bigotimes_k\Cplx^d$.
\begin{lstlisting}
In[1]:= Ket[0, 3]

Out[1]= {1, 0, 0}

In[2]:= KetFromDigits[{2, 1}, 3]

Out[2]= {0, 0, 0, 0, 0, 0, 0, 1, 0}
\end{lstlisting}
Because of the way the \Mathematica{} handles matrices and vectors kets are 
represented as one-dimensional lists. The inner product of two vectors can be 
calculated using the \Mathematica{} \lstinline!Dot! function. Outer product of 
vectors can be obtained using function \lstinline!KroneckerProduct!

\subsection{Density operators}
The most general state of a~quantum system is described by a~density operator.
In quantum mechanics a~density operator $\rho$ is defined as hermitian
($\rho=\rho^\dagger$) positive semi-definite ($\rho\geq 0$) trace one
($\tr{(\rho)}=1$) operator. When a basis is fixed the density operator can be
written in the form of a~matrix. 

\todo{Not really}
In \qi{} there are various ways to create a~density operator.

%\subsubsection{Entanglement}
%Entanglement is one of the most important phenomena in quantum information 
%theory. 
%We say that state $\rho$ is separable iff it can be written in the
%following form
%\begin{equation}
%\rho=\sum_{i=1}^{M} q_i\, \rho_i^A \otimes \rho_i^B,
%\end{equation}
%where $q_i>0$ and $\sum_{i=1}^{M} q_i=1$.
%A state that is not separable is called entangled. It is an open problem of
%great importance and under investigation, to decide if a~given quantum state 
%is
%entangled or not.
%
%\subsubsection{Subsystems}
%Given two states $\rho^A$, $\rho^B$ of two systems $A$ and $B$, the product 
%state
%$\rho^{AB}$ of the composed system is obtained by taking the Kronecker product
%of the states i.e. $\rho^{AB}=\rho^A\otimes\rho^B.$
%
%Let $[\rho^{AB}]_{kl}$ be a matrix representing a~quantum system composed of 
%two
%subsystems of dimensions $M$ and $N$. We want to index the matrix elements of
%$\rho$ using two double indices $[\rho^{AB}]_{\stackidx{m}{\mu}{n}{\nu}},$ so 
%that
%Latin indices correspond to the system $A$ and Greek indices correspond to the
%system $B$. The relation between indices is as follows $k=(m-1) N + \mu$, 
%$l=(n-1) N
%+ \nu$. 
%The partial trace with respect to system $B$ reads
%$\tr_B(\rho^{AB})=\sum_\mu \rho_{\stackidx{m}{\mu}{n}{\mu}}=\rho^A$, 
%and the partial trace with respect to system $A$ reads
%$\tr_A(\rho^{AB})=\sum_m \rho_{\stackidx{m}{\mu}{m}{\nu}}=\rho^B$.
%
%Given the state of the composed system $\rho^{AB}$ the state of subsystems can
%by found by the means of taking partial trace of $\rho^{AB}$ with respect to 
%one
%of the subsystems. It should noted that tracing-out is not a reversible
%operation, so in a~general case
%\begin{equation}
%    \rho^{AB}\neq \tr_A(\rho^{AB})\otimes\tr_B(\rho^{AB}).
%\end{equation}

\subsection{Completely positive trace-preserving maps (CPTP)}
We~say that an operation is physical if it transforms density operators into
density operators. Additionally we assume that physical operations are linear.
Therefore an operation $\Phi(\cdot)$ to be physical has to fulfil the following
set of conditions:
\begin{enumerate}
	\item For any operator $\rho$ its image under operation $\Phi$ has to have 
its
	trace and positivity preserved i.e. if 
	$\tr{(\rho)}=1, \rho\geq0, \rho=\rho^\dagger$ then
	$\tr{(\Phi(\rho))}=1, \Phi(\rho)\geq0, \Phi(\rho)=\Phi(\rho)^\dagger.$
	\item Operator $\Phi$ has to be linear:
	\begin{equation}
		\Phi\left(\sum_i p_i\rho_i \right)=\sum_i p_i \Phi\left(\rho_i \right).
	\end{equation}
	\item The extension of the operator $\Phi$ to any larger dimension that 
acts
	trivially on the extended system has to preserve positivity. 
	This feature is called complete positivity. It means that for all positive 
semi-definite
	$\rho,\xi\geq 0$ the following holds
	\begin{equation}
		(\Phi\otimes\1_{\dim{(\xi)}})\left(\rho\otimes\xi\right)
		=\Phi\left(\rho\right)\otimes\xi \geq 0.
	\end{equation}
\end{enumerate}
CPTP maps are often called quantum channels.

There are various ways to construct, represent and apply quantum channels in 
\qi{}: Kraus form, super-operator matrix and general function.

\subsubsection{General function}
Any function that fulfils criteria listed above is treated as physically 
allowed. 

\qi{} provides set of commonly used quantum channels:
\begin{itemize}
\item The trivial channel: \lstinline!IdentityChannel[n,\[Rho]]! 
$\rho=id_n(\rho)$

\item \lstinline!DepolarizingChannel[n,p,\[Rho]]! $\rho=id_n(\rho)$
\end{itemize}

%TransposeChannel::usage = "TransposeChannel[n,\[Rho]] - apply the 
%transposition operation to a n-dimensional density matrix \[Rho]. Note that 
%this operations is not completely positive.";
%
%



%HolevoWernerChannel::usage = "HolevoWernerChannel[n,p,\[Rho]] - apply the 
%Holeve-Werner channel, also known as transpose-depolarizing channel, with 
%parameter p acting to a n-dimensional input state \[Rho]. See also: 
%DepolarizingChannel.";

\subsubsection{Kraus form}
Any operator $\Phi$ that is completely positive and trace preserving can be
expressed in so called Kraus form \cite{BZ06}, which consists
of the finite set $\{E_k\}$ of Kraus operators -- matrices that fulfil the
completeness relation: $\sum_k {E_k}^\dagger E_k=\1$.
The image of state $\rho$ under 
the map $\Phi$ is given by 
\begin{equation}
    \Phi(\rho)=\sum_k E_k \rho {E_k}^\dagger.
\end{equation}

In \qi{} any list of matrices of equal dimensions can be treated as Kraus form.


\subsubsection{Super-operator matrix}



\begin{lstlisting}



ChannelToMatrix::usage = "ChannelToMatrix[E,d] returns matrix representation 
of a channel E acting on d-dimensional state space. First argument should be a 
pure function E such that E[\[Rho]] transforms input state according to the 
channel definition.";


GeneralizedPauliKraus::usage = "GeneralizedPauliKraus[d,P] - list of Kraus 
operators for d-dimensional generalized Pauli channel with the d-dimesnional 
matrix of parameters P. See: M. Hayashi, Quantum Information An Introduction, 
Springer 2006, Example 5.8, p. 126.";


ApplyKraus::usage = "ApplyKraus[ck,\[Rho]] - apply channel ck, given as a list 
of Kraus operators, to the input state \[Rho]. See also: ApplyUnitary, 
ApplyChannel.";


ApplyUnitary::usage = "ApplyUnitary[U,\[Rho]] - apply unitary a unitary matrix 
U to the input state \[Rho]. See also: ApplyKraus, ApplyChannel.";


ApplyChannel::usage = "ApplayChannel[f,\[Rho]] - apply channel f, given as a 
pure function, to the input state \[Rho]. See also: ApplyUnitary, ApplyKraus."


Superoperator::usage = "Superoperator[kl] returns matrix representation of 
quantum channel given as a list of Kraus operators. Superoperator[fun,dim] is 
just am alternative name for ChannelToMatrix[fun,dim] and returns matrix 
representation of quantum channel, given as a pure function, acting on 
dim-dimensional space. So Superoperator[DepolarizingChannel[2,p,#]&,2] and 
Superoperator[QubitDepolarizingKraus[p]] returns the same matrix. See also: 
ChannelToMatrix.";


DynamicalMatrix::usage = "Dynamical matrix of quantum channel given as a list 
of Kraus operators (DynamicalMatrix[ch]) or as a function fun action on 
dim-dimensional space (DynamicalMatrix[fun,dim]). See also: Superoperator, 
ChannelToMatrix.";


Jamiolkowski::usage = "Jamiolkowski[K] gives the image of the Jamiolkowski 
isomorphism for the channel given as the list of Karus operators K. 
Jamiolkowski[fun,dim] gives the image of the Jamiolkowski isomorphism for the 
channel given as a function fun action on dim-dimensional space. See also: 
Superoperator, ChannelToMatrix, DynamicalMatrix.";


TPChannelQ::usage = "Performs some checks on Kraus operators. Use this if you 
want to check if they represent quantum channel.";


ExtendKraus::usage = "ExtendKraus[ch,n] - produces n-fold tensor products of 
Kraus operators from the list ch.";


SuperoperatorToKraus::usage = "Finds Kraus operators for a given super 
operator";
\end{lstlisting}

\subsection{Measurement}
Quantum states cannot be observed directly. In the literature one considers two
main types of measurements: Von Neumann measurement and POVM (Positive Operator
Valued Measure) measurement. In this paper we use only Von Neumann
measurement but for the sake of completeness we also define POVM measurement.

The mathematical formulation of Von Neumann measurement is given by a map from 
a
set of projection operators to real numbers. 

Let us consider an orthogonal complete set of projection operators 
$P=\{P_i\}_{i=1}^N$ and
the set of real measurement outcomes $O=\{o_i\}_{i=1}^N$. Mapping $P\rightarrow
O$ is called Von Neumann measurement. Assuming the system is in the state 
$\rho$,
the probability $p_i$ of measuring outcome $o_i$ is given by the relation 
$p_i=\tr(P_i
\rho)$.

POVM measurement can be considered as a generalisation of Von Neumann
measurement. Let us take a set of positive operators $F=\{F_i\}_{i=1}^N$ such
that $\sum_{i=1}^N F_i=\1$ and the set of real measurement outcomes
$O=\{o_i\}_{i=1}^N$. Mapping $F\rightarrow O$ is called POVM measurement. Given
the system is in the state $\rho$, the probability $p_i$ of measuring outcome 
$o_i$
is given by the relation $p_i=\tr(F_i \rho)$.

%%%%%%%%%%%%%%%%%%%%%%%%%%%%%%%%%%%%%%%%%%%%%%%%%%%%%%%%%%%%%%%%%%%%%%%%%%%%%%%%
\section{Concluding remarks}\label{sec:comclude}
%%%%%%%%%%%%%%%%%%%%%%%%%%%%%%%%%%%%%%%%%%%%%%%%%%%%%%%%%%%%%%%%%%%%%%%%%%%%%%%%
Package \qi\ provide set of functions which aims to simplify the task of
exploring the space of quantum states and understanding quantum operations.

%%%%%%%%%%%%%%%%%%%%%%%%%%%%%%%%%%%%%%%%%%%%%%%%%%%%%%%%%%%%%%%%%%%%%%%%%%%%%%%%
\section*{Acknowledgements}
%%%%%%%%%%%%%%%%%%%%%%%%%%%%%%%%%%%%%%%%%%%%%%%%%%%%%%%%%%%%%%%%%%%%%%%%%%%%%%%%
We acknowledge the financial support by Polish Research Network LFPPI.

\appendix
%%%%%%%%%%%%%%%%%%%%%%%%%%%%%%%%%%%%%%%%%%%%%%%%%%%%%%%%%%%%%%%%%%%%%%%%%%%%%%%%
\section{List of provided functions}
%%%%%%%%%%%%%%%%%%%%%%%%%%%%%%%%%%%%%%%%%%%%%%%%%%%%%%%%%%%%%%%%%%%%%%%%%%%%%%%%

%\input{qi_functions}

\bibliography{qi}
\bibliographystyle{elsarticle-num}

\end{document}
\endinput
%%
%% End of file `elsarticle-template-num.tex'.
