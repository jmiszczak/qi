% v. 0.2 - 11/08/2009 - Jarek - extended introductions and bibliography
% v. 0.21 - 12/08/2009 - Jarek - preliminary sections
% v. 0.22 - 13/08/2009 - Jarek - added in intro and sec. 2
% v. 0.3 - 17/11/2009 - Jarek - added an appendix with a list of functions
% v. 0.31 - 18/11/2009 - Jarek - sections on states and operations
% v. 0.32 - 19/11/2009 - Jarek - improved abstact, applied CPC template 
% v. 0.33 - 24/09/2010 - Jarek - update basic info, sync with 0.3.18 version 
\documentclass{elsart}

%% This list environment is used for the references in the
%% Program Summary
%%
\newcounter{bla}
\newenvironment{refnummer}{%
\list{[\arabic{bla}]}%
{\usecounter{bla}%
 \setlength{\itemindent}{0pt}%
 \setlength{\topsep}{0pt}%
 \setlength{\itemsep}{0pt}%
 \setlength{\labelsep}{2pt}%
 \setlength{\listparindent}{0pt}%
 \settowidth{\labelwidth}{[9]}%
 \setlength{\leftmargin}{\labelwidth}%
 \addtolength{\leftmargin}{\labelsep}%
 \setlength{\rightmargin}{0pt}}}
 {\endlist}

%%%%%%%%%%%%%%%%%%%%%%%%%%%%%%%%%%%%%%%%%%%%%%%%%%%%%%%%%%%%%%%%%%%%%%%%%%%%%%%%
\usepackage{amsmath,amssymb}
\usepackage{dsfont}
\newcommand{\ket}[1]{\ensuremath{|#1\rangle}}
\newcommand{\bra}[1]{\ensuremath{\langle#1|}}
\newcommand{\Mathematica}{\emph{Mathematica}}
\newcommand{\wek}{\mathbf{vec}}
\newcommand{\res}{\mathbf{res}}
\newcommand{\1}{{\rm 1\hspace{-0.9mm}l}}
\newcommand{\Id}{\1}
\newcommand{\SWAP}{\ensuremath{\mathrm{SWAP}}}
\newcommand{\tr}{\mathrm{tr}}
\newcommand{\M}{\ensuremath{\mathbb{M}}}
\newcommand{\qi}{QI}
\newcommand{\code}[1]{\texttt{\small #1}}
\newcommand{\HS}[1]{\ensuremath{\mathcal{#1}}} % Hilbert space
\newcommand{\Cplx}{\ensuremath{\mathbb{C}}}
\newcommand{\eg}{\emph{eg.}}
\newcommand{\ie}{\emph{ie.}}
%\newcommand{\etal}{\emph{et al.}}
\newcommand{\Prob}[1]{\ensuremath{\mathrm{P}(#1)}}
\newcommand{\Observ}[1]{\ensuremath{#1}}
\newcommand{\Spectrum}[1]{\ensuremath{\sigma(#1)}}
\newcommand{\Spec}[1]{\Spectrum{#1}}
\newcommand{\ketbra}[2]{\ensuremath{\ket{#1}\bra{#2}}}
\newcommand{\proj}[1]{\ensuremath{\ketbra{#1}{#1}}}
\newcommand{\Proj}[1]{\proj{#1}}
\newcommand{\iner}[2]{\braket{#1}{#2}}
\newcommand{\Iner}[2]{\iner{#1}{#2}}
\newcommand{\braket}[2]{\ensuremath{\langle#1|#2\rangle}}
\newcommand{\Tr}[2][]{\ensuremath{\tr_{#1}{#2}}}
\newcommand{\id}{\mathds{1}}
\newcommand{\Space}[1]{\mathcal{#1}}
\newcommand{\SetOfStates}[1]{\ensuremath{\mathcal{S}(#1)}}
\newcommand{\States}[1]{\SetOfStates{#1}}
\newcommand{\Real}{\ensuremath{\mathds{R}}}
\newcommand{\N}{\ensuremath{\mathds{N}}}
\newcommand{\Z}{\ensuremath{\mathds{Z}}}
\newcommand{\set}[2]{\ensuremath{\left\{#1|#2\right\}}}
\newcommand{\re}[1]{\ensuremath{\mathrm{Re}\left(#1\right)}}
\newcommand{\im}[1]{\ensuremath{\mathrm{Im}\left(#1\right)}}
\newcommand{\Group}[2]{\ensuremath{(#1,#2)}}
\newcommand{\ord}[2]{\mathrm{ord}_#2(#1)}
\newcommand{\Var}[1]{\ensuremath{\mathrm{Var}(#1)}}
\newcommand{\halmos}{\newline\vspace{3mm}\hfill $\Box$}
\newcommand{\proof}{\noindent {\it Proof.\ }}
\newtheorem{theorem}{Theorem}

\usepackage{xcolor}
\newcommand{\todo}[1]{\textcolor{red}{\bf TODO: #1}}
%%%%%%%%%%%%%%%%%%%%%%%%%%%%%%%%%%%%%%%%%%%%%%%%%%%%%%%%%%%%%%%%%%%%%%%%%%%%%%%%

\journal{Computer Physics Communications}

\begin{document}

\begin{frontmatter}

%% Title, authors and addresses

%% use the tnoteref command within \title for footnotes;
%% use the tnotetext command for theassociated footnote;
%% use the fnref command within \author or \address for footnotes;
%% use the fntext command for theassociated footnote;
%% use the corref command within \author for corresponding author footnotes;
%% use the cortext command for theassociated footnote;
%% use the ead command for the email address,
%% and the form \ead[url] for the home page:
%% \title{Title\tnoteref{label1}}
%% \tnotetext[label1]{}
%% \author{Name\corref{cor1}\fnref{label2}}
%% \ead{email address}
%% \ead[url]{home page}
%% \fntext[label2]{}
%% \cortext[cor1]{}
%% \address{Address\fnref{label3}}
%% \fntext[label3]{}

\title{\qi: A package for the analysis of quantum states and operations in
\Mathematica}

\date{24/09/2009 (v. 0.33)}

\author{J.~A.~Miszczak\thanksref{author}},
\author{P.~Gawron},
\author{Z.~Pucha{\l}a}

\thanks[author]{Corresponding author}

\address{Institute of Theoretical and Applied Informatics, Polish Academy of 
Sciences, Ba{\l}tycka 5, 44-100 Gliwice, Poland}

\begin{abstract}
QI is a package of functions for the \Mathematica\ computer algebra system,
which provides a framework for the analysis of quantum states and quantum
operations. In contrast to many available packages for symbolic and numerical
simulation of quantum computation presented package is focused on geometrical
aspects of quantum information theory. In particular \qi provides
parametrization of quantum states and selected families of quantum operation,
function for constructing composite quantum operations and methods for the
generation and analysis of random quantum operations. Also basic structures are
provided including the construction of ket vectors, basic unitary gates, random
states and unitaries, distance measures between quantum states and the selected
methods for the analysis of separability in quantum systems.

\begin{flushleft}
  %Insert your suggested PACS number here
%quantum information, symbolic computation (computer algebra) 
PACS: 03.67.-a; 02.70.Wz.
\end{flushleft}

\begin{keyword}
  % Please give some freely chosen keywords that we can use in a
  % cumulative keyword index.
Quantum states; Quantum operations; Partial operations.
\end{keyword}
\end{abstract}

\end{frontmatter}
%%%%%%%%%%%%%%%%%%%%%%%%%%%%%%%%%%%%%%%%%%%%%%%%%%%%%%%%%%%%%%%%%%%%%%%%%%%%%%%%
{\bf PROGRAM SUMMARY}
  %Delete as appropriate.

\begin{small}
\noindent
{\em Manuscript Title:} \qi: A package for the analysis of quantum states and operations in
\Mathematica\\
{\em Authors:} J.A.~Miszczak, P.~Gawron, Z.~Pucha{\l}a \\
{\em Program Title:} QI \\
{\em Journal Reference:}                                      \\
  %Leave blank, supplied by Elsevier.
{\em Catalogue identifier:}                                   \\
  %Leave blank, supplied by Elsevier.
{\em Licensing provisions:} GPLv3 \\
{\em Programming language:} Mathematica 7\\
{\em Computer:} Any computer supporting Mathematica 7\\
  %Computer(s) for which program has been designed.
{\em Operating system:} Any operating system capable of running Mathematica 7, \eg\ GNU/Linux, MacOS X, FreeBSD, Microsoft Windows XP\\
  %Operating system(s) for which program has been designed.
{\em RAM:} bytes                                              \\
  %RAM in bytes required to execute program with typical data.
{\em Number of processors used:}                              \\
  %If more than one processor.
{\em Supplementary material:}                                 \\
  % Fill in if necessary, otherwise leave out.
{\em Keywords:} quantum states, quantum operations, partial operations  \\
  % Please give some freely chosen keywords that we can use in a
  % cumulative keyword index.
{\em PACS:} 03.67.-a, 02.70.Wz.\\
  % see http://www.aip.org/pacs/pacs.html
{\em Classification:} 4.15 \\
  %Classify using CPC Program Library Subject Index, see (
  % http://cpc.cs.qub.ac.uk/subjectIndex/SUBJECT_index.html)
  %e.g. 4.4 Feynman diagrams, 5 Computer Algebra.
{\em External routines/libraries:}                                      \\
  % Fill in if necessary, otherwise leave out.
{\em Subprograms used:}                                       \\
  %Fill in if necessary, otherwise leave out.

{\em Nature of problem:}\\
  %Describe the nature of the problem here.
  Construction of composed quantum operations, analysis of quantum states and
  operations.
   \\
{\em Solution method:}\\
  %Describe the method solution here.
  A package of functions for \Mathematica\ computer algebra system.
   \\
{\em Restrictions:}\\
  %Describe any restrictions on the complexity of the problem here.
   \\
{\em Unusual features:}\\
  %Describe any unusual features of the program/problem here.
   \\
{\em Additional comments:}\\
  %Provide any additional comments here.
   \\
{\em Running time:}\\
  %Give an indication of the typical running time here.
   \\

\end{small}

\newpage

% In program descriptions the main text of the paper is listed under
% the heading LONG WRITE-UP.

\hspace{1pc}
{\bf LONG WRITE-UP}
%%%%%%%%%%%%%%%%%%%%%%%%%%%%%%%%%%%%%%%%%%%%%%%%%%%%%%%%%%%%%%%%%%%%%%%%%%%%%%%%


\tableofcontents

%% \linenumbers

%% main text
%%%%%%%%%%%%%%%%%%%%%%%%%%%%%%%%%%%%%%%%%%%%%%%%%%%%%%%%%%%%%%%%%%%%%%%%%%%%%%%%
\section{Introduction}\label{sec:intro}
%%%%%%%%%%%%%%%%%%%%%%%%%%%%%%%%%%%%%%%%%%%%%%%%%%%%%%%%%%%%%%%%%%%%%%%%%%%%%%%%
Quantum information theory aims to harness the behavior of quantum mechanical
objects to store, transfer and process information~\cite{hayashi}. This behavior
is in many cases very different from the one we observe in classical world.
Quantum algorithms and protocols take advantage of superposition of states and
require the presence of entangled states. Both phenomena arise from the rich
structure of the space of quantum states. Hence, to explore the capabilities of
quantum information processing, one needs to fully understand this
space~\cite{BZ06}. 

Quantum mechanics provides us also with much larger than in classical case space
of allowed operations which can be used to manipulate quantum
states~\cite{hayashi,BZ06}. Recent results concerning additivity
problems~\cite{hastings09superadditivity} show that we are far from full
understanding the nature of quantum channels. Exploring the space of quantum
operations is fascinating, but cumbersome task.

%One of the motivations behind quantum information theory was an observation that
%the simulation of quantum systems on classical machine requires exponential 
%resources.

We present a~package of functions developed for \Mathematica\ 7.0 which aims to
simplify the analysis of quantum states and quantum operations. The package was 
developed in simplicity in mind and thus it uses only basic data structures
available in \Mathematica. This allows to relatively easily port implemented
functions to other scientific software systems. Also, in contrast to most
quantum computing packages available~\cite{qdensity,qucalc,quantum2}, \qi\ is
not aimed to provide tool for simulating quantum algorithms and protocols. We
rather focus on the analysis of quantum states used in those protocols and
quantum channels, which are used to describe allowed physical operations. The
main goal of presented package is to provide basic mathematical tools useful for
studding quantum stats and quantum channels.

This paper is organized as follows. In Section~\ref{sec:qi-intro} we review
basic notion of quantum density matrices and quantum channels representing
allowed physical transformations of density matrices. Section~\ref{sec:over}
presents an overview of functionality provided by \qi\ package and describes
some operations used as building blocks in specialized functions.
Sections~\ref{sec:states} and \ref{sec:channels} provide detailed description of
functions implemented in~\qi. Finally Section~\ref{sec:comclude} provides some
concluding remarks.

%%%%%%%%%%%%%%%%%%%%%%%%%%%%%%%%%%%%%%%%%%%%%%%%%%%%%%%%%%%%%%%%%%%%%%%%%%%%%%%%
\section{Quantum states and operations}\label{sec:qi-intro}
%%%%%%%%%%%%%%%%%%%%%%%%%%%%%%%%%%%%%%%%%%%%%%%%%%%%%%%%%%%%%%%%%%%%%%%%%%%%%%%%
Let $\HS{H}$ be a~separable, complex Hilbert space used to described the system
in question. In quantum information theory we deal mainly with
finite-dimensional Hilbert spaces, so usually we have to are in situation where
$\HS{H}=\Cplx^n$. State of the system is described by the density matrix, \ie\
operator $\rho:\HS{H}\rightarrow\HS{H}$, which is positive ($\rho\geq0$) and
normalized ($\tr{\rho}=1$). 

%%%%%%%%%%%%%%%%%%%%%%%%%%%%%%%%%%%%%%%%%%%%%%%%%%%%%%%%%%%%%%%%%%%%%%%%%%%%%%%%
\subsection{Pure and mixed states}
%%%%%%%%%%%%%%%%%%%%%%%%%%%%%%%%%%%%%%%%%%%%%%%%%%%%%%%%%%%%%%%%%%%%%%%%%%%%%%%%
Let $\HS{H}$ be a~separable, complex Hilbert space\index{Hilbert space} used
to described the system in question. We use Dirac notation
\cite{dirac58principles} for inner product in $\HS{H}$
\begin{equation}
  \Iner{\psi}{\phi},
\label{eqn:dirac-iner}
\end{equation}
where $\ket{\psi},\ket{\phi}\in\HS{H}$ and $\bra{\phi} = \ket{\phi}^\star$ 
is a~complex conjugate of $\ket{\phi}$. Symbol $\ket{\psi}\bra{\phi}$ denotes
the operator of rank one (\ie\ a projection operator) which acts on a vector
$\ket{\alpha}\in\HS{H}$ as
\begin{equation}
\left(\ket{\psi}\bra{\phi}\right)\ket{\alpha} = \Iner{\phi}{\alpha}\ket{\psi}.
\label{eqn:dirac-proj}
\end{equation}

In quantum information theory we deal mainly with finite-dimensional Hilbert
spaces.

%%%%%%%%%%%%%%%%%%%%%%%%%%%%%%%%%%%%%%%%
\subsection{Observables}
%%%%%%%%%%%%%%%%%%%%%%%%%%%%%%%%%%%%%%%%
The term \emph{observable}\index{observable} is used to describe a physical
quantity of a system which can be observed and measured. Observables are
quantum mechanical analogous of the \emph{random variables}\index{random
variable} from classical mechanics.

In quantum mechanics observables are described by self-adjoint operators,
\ie\ linear functions $X:\HS{H}\mapsto\HS{H}$ such that
\begin{equation}
\Iner{X\psi}{\phi}=\Iner{\psi}{X^\star\phi},
\label{eqn:selfadj}
\end{equation}
for any $\ket{\phi},\ket{\psi}\in\HS{H}$. An important property of self-adjoint
operators is expressed by the following theorem.

\begin{theorem}[Spectral decomposition]
Every self-adjoint operator $\Observ{A}$ can be decomposed according to the
formula
\begin{equation}
\Observ{A} = \sum_{\lambda_i\in\Spec{\Observ{A}}} \lambda_i\proj{x_i}
\end{equation}
with $\sum_i\Proj{x_i}=\Id$, where $\ket{x_i}$ are the eigenvectors of
$\Observ{A}$ with corresponding eigenvalues $\lambda_i$.
\end{theorem}
Here $\sigma(A)$ denotes the spectrum of the operator $A$. The mapping
$\mu:\lambda_i\mapsto \Proj{x_i}$ is called \emph{spectral
measure} \cite{mlak}.\index{spectral measure}

Spectral decomposition can be used to define functions on the space of
self-adjoint operators. If $f:\Real\mapsto\Cplx$ is any function, one can define
$f(\Observ{A})$ as
\begin{equation}
 f(A) = \sum_{\lambda_i\in\Spec{\Observ{A}}} f(\lambda_i)\proj{x_i}.
\end{equation}
For example, if we take $A=NOT$, we have
\begin{equation}
 NOT =  +1\left(\begin{array}{cc}
 		\frac{1}{2} & \frac{1}{2}\\
 		\frac{1}{2} & \frac{1}{2}
        \end{array}\right)
		-1 \left(\begin{array}{cc}
                          \frac{1}{2} & -\frac{1}{2}\\
                          -\frac{1}{2} & \frac{1}{2}
                         \end{array}\right),
\end{equation}
and square root of $NOT$ can be easily calculated as
\begin{equation}
 \sqrt{NOT} = \sqrt{+1}\left(\begin{array}{cc}
 		\frac{1}{2} & \frac{1}{2}\\
 		\frac{1}{2} & \frac{1}{2}
        \end{array}\right)
		+\sqrt{-1} \left(\begin{array}{cc}
                          \frac{1}{2} & -\frac{1}{2}\\
                          -\frac{1}{2} & \frac{1}{2}
                         \end{array}\right).
\end{equation}

%%%%%%%%%%%%%%%%%%%%%%%%%%%%%%%%%%%%%%%%
\subsection{States}
%%%%%%%%%%%%%%%%%%%%%%%%%%%%%%%%%%%%%%%%
In quantum mechanics the state of the system is described by the density matrix,
\ie\ Hermitian operator $\rho:\HS{H}\rightarrow\HS{H}$, which is
\begin{equation}
 \rho\geq0\ \ \mathrm{(positive)}
\end{equation}
and
\begin{equation}
 \tr{\rho}=1\ \ \mathrm{(normalized)}.
\end{equation}
Density matrix in the analogue of the classical \emph{probability
distribution}. \index{probability distribution}

The set of states $\SetOfStates{\mathcal{H}}$ is a~convex set and thus every
$\rho\in\SetOfStates{\mathcal{H}}$ can be represented as a~convex combination
\begin{equation}
\rho=\sum_{i}p_i\sigma_i,
\end{equation}
with $\sigma_i\in\SetOfStates{\mathcal{H}},\ \sum_{i}p_i=1$. 

If the operator $\rho$ additionally fulfils the condition
\begin{equation}
 \rho^2 = \rho
\end{equation}
\ie\ $\rho$ is a~projection operator, the state described by $\rho$ is
said to be \emph{pure}.\index{pure state} In this case density operator posesses
only one eigenvector $\ket{\alpha}$, which can be used to describe the state.
Therefore, in the case of pure states, we can write
\begin{equation}
  \rho = \proj{\alpha}.
\end{equation}

In particular, if $\HS{H}=\Cplx^2$ we say that $\ket{\psi}\in\HS{H}$ describes
the state of a qubit. In this case
\begin{equation}
 \ket{\psi} = \alpha \ket{0} + \beta \ket{1},
\end{equation}
where $\alpha^2+\beta^2=1$ and $\ket{0}, \ket{1}\in \Cplx^2$ form a base
$\Cplx^2$. 

%%%%%%%%%%%%%%%%%%%%%%%%%%%%%%%%%%%%%%%%
\subsection{Unitary evolution}\label{sec:math-unit-evol}
%%%%%%%%%%%%%%%%%%%%%%%%%%%%%%%%%%%%%%%%
Let us assume that the described system is isolated during the time of
evolution. If initially the system is in a pure state $\ket{\psi}$, then the
evolution of the system is given by some unitary operator $U$. The action of the
operator $U$ on the initial state is given by a formula
\begin{equation}
 \ket{\psi} \mapsto U \ket{\psi}.
\end{equation}
If the initial state of the system is mixed and given by the density matrix
$\rho$, then the final state is given by
\begin{equation}\label{eqn:evol-mixed}
 \rho \mapsto U\rho U^\dagger.
\end{equation}

However, in many situations it is impossible to avoid the interaction of the
systems with an environment.


%%%%%%%%%%%%%%%%%%%%%%%%%%%%%%%%%%%%%%%%%%%%%%%%%%%%%%%%%%%%%%%%%%%%%%%%%%%%%%%%
\subsection{Quantum processes}
%%%%%%%%%%%%%%%%%%%%%%%%%%%%%%%%%%%%%%%%%%%%%%%%%%%%%%%%%%%%%%%%%%%%%%%%%%%%%%%%
The most general form of the evolution of a quantum system is given in terms of
quantum channels. In this paper we consider quantum channels which are 
Completely Positive Trace Preserving (CP-TP) maps. 

In order an map $\Phi$ to be a CP-TP map it has to fulfill set of the following
conditions:
\begin{enumerate}
\item It has to preserve trace, positivity and hermiticity, \ie\
  \begin{equation}
	  \Tr{\Phi(\rho)}=1, \Phi(\rho)\geq0\ \mathrm{and\ } \Phi(\rho)=\Phi(\rho)^\dagger.
  \end{equation}
\item It has to be linear
  \begin{equation}
	  \Phi\left(\sum_i p_i\rho_i \right)=\sum_i p_i \Phi\left(\rho_i \right).
  \end{equation}
\item Finally it has to be \emph{completely positive}, \ie\ for $\rho^{(n)} \geq
  0$ we require that
  \begin{eqnarray}
	   (\Phi\otimes\id_n) \rho^{(n)} \geq 0,\ n\in N,
  \end{eqnarray}
  where $\rho^{(n)}$ is an element of an appropriate space of 
  states.
\end{enumerate}
These conditions are required for $\Phi$ to preserve the set of quantum states.

In the most general case quantum evolution is described by a superoperator
$\Phi$, acting on $\mathcal{M}_{N}$, which can be expressed in Kraus form
\cite{BZ06,hayashi}
\begin{equation}
\Phi(\rho)=\sum_k E_k^{\phantom{\dagger}} \rho E_k^\dagger,
\end{equation}
where $\sum_k E_k^\dagger E_k^{\phantom{\dagger}}=\id$.

Alternatively quantum operations an be represented by a \emph{superoperator
matrix} $M_\Phi$. The superoperator matrix is a~representation of linear
operator in the canonical basis. The following formula allows to transform set
of Kraus operators $\{E_k\}$ into superoperator matrix $M_\Phi$~\cite[Ch.
10]{BZ06}
\begin{equation} \label{r:Mphi}
	M_\Phi=\sum_{k=1}^{N^2} E_k \otimes E^\ast_k, 
\end{equation}
where $N = \dim(E_k)$ and `$\ast$' denotes element-wise complex conjugation.

The dynamical matrix for the operations $\Phi$ is defined as $D_\Phi=M_\Phi^R$,
where `${}^R$' denotes a \emph{reshuffling} operation~\cite{BZ06}. The dynamical matrix
for the trace preserving operation acting on $N$-dimensional system is
an~$N^2\times N^2$ positive defined matrix with trace~$N$. We can introduce
natural correspondence between such matrices and density matrices on $N^2$ by
normalizing $D_\Phi$. Such a correspondence is known as \emph{Jamio{\l}kowski
isomorphism}~\cite{jamiolkowski72linear,zyczkowski04duality}.

Let $\Phi$ be a~completely positive trace preserving map acting on density
matrices. We define Jamio\l{}kowski matrix of $\Phi$ as
\begin{equation} \label{r:jam}
 \rho_\Phi = \frac{1}{N}D_\Phi. 
\end{equation}

Jamio\l{}kowski matrix has the same mathematical properties as a quantum state
\ie{} it is a semi-definite positive matrix with trace equal to one. It is
sometimes referred to as \emph{Jamio\l{}kowski state matrix}.



%%%%%%%%%%%%%%%%%%%%%%%%%%%%%%%%%%%%%%%%%%%%%%%%%%%%%%%%%%%%%%%%%%%%%%%%%%%%%%%%
\section{Overview of \qi}\label{sec:over}
%%%%%%%%%%%%%%%%%%%%%%%%%%%%%%%%%%%%%%%%%%%%%%%%%%%%%%%%%%%%%%%%%%%%%%%%%%%%%%%%
\todo{troche o zalozeniach projektowych czyli dlaczego \qi\ jest inne niz
pozostale pakiety}

%%%%%%%%%%%%%%%%%%%%%%%%%%%%%%%%%%%%%%%%%%%%%%%%%%%%%%%%%%%%%%%%%%%%%%%%%%%%%%%%
\subsection{Design principles}
%%%%%%%%%%%%%%%%%%%%%%%%%%%%%%%%%%%%%%%%%%%%%%%%%%%%%%%%%%%%%%%%%%%%%%%%%%%%%%%%

\newcounter{principle}
\begin{list}{\textbf{P\arabic{principle}}}{\usecounter{principle}}
\item Functions are as simple as possible -- functions have minimum reasonable
number of arguments and implement small pieces of functionality. For example in
order to get partial trace on few subsystems user needs to perform permutation
and then apply partial trace 
\item User knows what she is doing -- functions implemented in \qi\ do not
validate the input and user is expected to provide reasonable data.
\item Only basic \Mathematica\ structures are used -- functions operator on
plain \Mathematica\ lists and return lists.
\item Some data are used more often then other -- \qi\ predefines some commonly
used matrices, for example matrices for \SWAP\ operation and Pauli matrices for
small dimensions.
\end{list}

%%%%%%%%%%%%%%%%%%%%%%%%%%%%%%%%%%%%%%%%%%%%%%%%%%%%%%%%%%%%%%%%%%%%%%%%%%%%%%%%
\subsection{Basic algebraic operations}
%%%%%%%%%%%%%%%%%%%%%%%%%%%%%%%%%%%%%%%%%%%%%%%%%%%%%%%%%%%%%%%%%%%%%%%%%%%%%%%%

%%%%%%%%%%%%%%%%%%%%%%%%%%%%%%%%%%%%%%%%%%%%%%%%%%%%%%%%%%%%%%%%%%%%%%%%%%%%%%%%
\subsubsection{Kronecker product}
%%%%%%%%%%%%%%%%%%%%%%%%%%%%%%%%%%%%%%%%%%%%%%%%%%%%%%%%%%%%%%%%%%%%%%%%%%%%%%%%
Version 6 of \Mathematica\ introduced \code{KroneckerProduct} function.

%%%%%%%%%%%%%%%%%%%%%%%%%%%%%%%%%%%%%%%%%%%%%%%%%%%%%%%%%%%%%%%%%%%%%%%%%%%%%%%%
\subsection{Comonly used matrices}
%%%%%%%%%%%%%%%%%%%%%%%%%%%%%%%%%%%%%%%%%%%%%%%%%%%%%%%%%%%%%%%%%%%%%%%%%%%%%%%%
\qi\ predefines set of commonly used matrices. Among avaible matrices one can
find
\begin{itemize}
\item Pauli matrices
\begin{equation}
\left(
\begin{smallmatrix}
 0 & 1 \\
 1 & 0
\end{smallmatrix}
\right),\left(
\begin{smallmatrix}
 0 & -i \\
 i & 0
\end{smallmatrix}
\right),\left(
\begin{smallmatrix}
 1 & 0 \\
 0 & -1
\end{smallmatrix}
\right).
\end{equation}

\item Gell-Mann matrices
\begin{equation}
\left(
\begin{smallmatrix}
 0 & 1 & 0 \\
 1 & 0 & 0 \\
 0 & 0 & 0
\end{smallmatrix}
\right),\left(
\begin{smallmatrix}
 0 & -i & 0 \\
 i & 0 & 0 \\
 0 & 0 & 0
\end{smallmatrix}
\right),
\left(
\begin{smallmatrix}
 1 & 0 & 0 \\
 0 & -1 & 0 \\
 0 & 0 & 0
\end{smallmatrix}
\right),
\left(
\begin{smallmatrix}
 0 & 0 & 1 \\
 0 & 0 & 0 \\
 1 & 0 & 0
\end{smallmatrix}
\right),
\left(
\begin{smallmatrix}
 0 & 0 & -i \\
 0 & 0 & 0 \\
 i & 0 & 0
\end{smallmatrix}
\right),
\left(
\begin{smallmatrix}
 0 & 0 & 0 \\
 0 & 0 & 1 \\
 0 & 1 & 0
\end{smallmatrix}
\right),
\left(
\begin{smallmatrix}
 0 & 0 & 0 \\
 0 & 0 & -i \\
 0 & i & 0
\end{smallmatrix}
\right),
\frac{1}{\sqrt{3}}\left(
\begin{smallmatrix}
 1 & 0 & 0 \\
 0 & 1 & 0 \\
 0 & 0 & -2
\end{smallmatrix}
\right).
\end{equation}
\item Hadamard matrix 
\begin{equation}
\left(
\begin{smallmatrix}
 \frac{1}{\sqrt{2}} & \frac{1}{\sqrt{2}} \\
 \frac{1}{\sqrt{2}} & -\frac{1}{\sqrt{2}}
\end{smallmatrix}
\right)
\end{equation}
\end{itemize}

%%%%%%%%%%%%%%%%%%%%%%%%%%%%%%%%%%%%%%%%%%%%%%%%%%%%%%%%%%%%%%%%%%%%%%%%%%%%%%%%
\section{Quantum states in \qi}\label{sec:states}
%%%%%%%%%%%%%%%%%%%%%%%%%%%%%%%%%%%%%%%%%%%%%%%%%%%%%%%%%%%%%%%%%%%%%%%%%%%%%%%%

%%%%%%%%%%%%%%%%%%%%%%%%%%%%%%%%%%%%%%%%%%%%%%%%%%%%%%%%%%%%%%%%%%%%%%%%%%%%%%%%
\subsection{Probability distributions}
%%%%%%%%%%%%%%%%%%%%%%%%%%%%%%%%%%%%%%%%%%%%%%%%%%%%%%%%%%%%%%%%%%%%%%%%%%%%%%%%
State in quantum mechanics is a generalization of classical probability
distribution.

%%%%%%%%%%%%%%%%%%%%%%%%%%%%%%%%%%%%%%%%%%%%%%%%%%%%%%%%%%%%%%%%%%%%%%%%%%%%%%%%
\subsection{One-qubit states}
%%%%%%%%%%%%%%%%%%%%%%%%%%%%%%%%%%%%%%%%%%%%%%%%%%%%%%%%%%%%%%%%%%%%%%%%%%%%%%%%

%%%%%%%%%%%%%%%%%%%%%%%%%%%%%%%%%%%%%%%%%%%%%%%%%%%%%%%%%%%%%%%%%%%%%%%%%%%%%%%%
\subsection{States in higher dimension}
%%%%%%%%%%%%%%%%%%%%%%%%%%%%%%%%%%%%%%%%%%%%%%%%%%%%%%%%%%%%%%%%%%%%%%%%%%%%%%%%
\qi\ provides function for cosntructing pure state in arbitrary dimension

Parametrization of pure states in higher dimensions is provided according to
\cite{tilma02generalized}.

%%%%%%%%%%%%%%%%%%%%%%%%%%%%%%%%%%%%%%%%%%%%%%%%%%%%%%%%%%%%%%%%%%%%%%%%%%%%%%%%
\subsection{Special states}
%%%%%%%%%%%%%%%%%%%%%%%%%%%%%%%%%%%%%%%%%%%%%%%%%%%%%%%%%%%%%%%%%%%%%%%%%%%%%%%%
\begin{itemize}
\item maximally entangled states
\item maximally mixed states
\item Werner states
\item Isotropic states
\end{itemize}

\todo{zapoznac sie ze stanami horodeckich z bound entanglement}
%%%%%%%%%%%%%%%%%%%%%%%%%%%%%%%%%%%%%%%%%%%%%%%%%%%%%%%%%%%%%%%%%%%%%%%%%%%%%%%%
\subsection{Distance measures}
%%%%%%%%%%%%%%%%%%%%%%%%%%%%%%%%%%%%%%%%%%%%%%%%%%%%%%%%%%%%%%%%%%%%%%%%%%%%%%%%


%%%%%%%%%%%%%%%%%%%%%%%%%%%%%%%%%%%%%%%%%%%%%%%%%%%%%%%%%%%%%%%%%%%%%%%%%%%%%%%%
\section{Quantum operations in \qi}\label{sec:channels}
%%%%%%%%%%%%%%%%%%%%%%%%%%%%%%%%%%%%%%%%%%%%%%%%%%%%%%%%%%%%%%%%%%%%%%%%%%%%%%%%

%%%%%%%%%%%%%%%%%%%%%%%%%%%%%%%%%%%%%%%%%%%%%%%%%%%%%%%%%%%%%%%%%%%%%%%%%%%%%%%%
\subsection{Representation of quantum operations}
%%%%%%%%%%%%%%%%%%%%%%%%%%%%%%%%%%%%%%%%%%%%%%%%%%%%%%%%%%%%%%%%%%%%%%%%%%%%%%%%

%%%%%%%%%%%%%%%%%%%%%%%%%%%%%%%%%%%%%%%%%%%%%%%%%%%%%%%%%%%%%%%%%%%%%%%%%%%%%%%%
\subsection{Random quantum operations}
%%%%%%%%%%%%%%%%%%%%%%%%%%%%%%%%%%%%%%%%%%%%%%%%%%%%%%%%%%%%%%%%%%%%%%%%%%%%%%%%
\cite{Bruzda2009320}

%%%%%%%%%%%%%%%%%%%%%%%%%%%%%%%%%%%%%%%%%%%%%%%%%%%%%%%%%%%%%%%%%%%%%%%%%%%%%%%%
\subsection{Partial operations}
%%%%%%%%%%%%%%%%%%%%%%%%%%%%%%%%%%%%%%%%%%%%%%%%%%%%%%%%%%%%%%%%%%%%%%%%%%%%%%%%
\qi\ implements partial operations using general method for constructing
channels acting on subsystems. This method is based on formula
\begin{equation}\label{eqn:def-tensor}
(\Phi\otimes\Id)(\rho) = 
\left(\res^{-1}\left(M_\Phi\res\left(\rho^R\right)\right)\right)^R
\end{equation}
where ${}^R$ denotes the reshuffling operation and $M_\Phi$ denotes the matrix
of the linear map $\Phi$
\begin{equation}
M_\Phi = \tr \epsilon_i \Phi(\epsilon_j),
\end{equation}
where $\{\epsilon_i\}_i=1,\ldots,n^2$ is a canonical basis in $\M_n$. In the 
Eq.~\ref{eqn:def-tensor} operation $\Phi$ is applied to the first subsystem 
only.

\todo{to jest do przepisania}
%%%%%%%%%%%%%%%%%%%%%%%%%%%%%%%%%%%%%%%%%%%%%%%%%%%%%%%%%%%%%%%%%%%%%%%%%%%%%%%%
\subsubsection{Partial transposition}
%%%%%%%%%%%%%%%%%%%%%%%%%%%%%%%%%%%%%%%%%%%%%%%%%%%%%%%%%%%%%%%%%%%%%%%%%%%%%%%%
In the case of transposition $M_\Phi$ is equivalent to \SWAP. For example in the
case of $4$-dimensional density matrix
\begin{equation}
X=\left(
\begin{array}{cccc}
 \alpha_{1,1} & \alpha_{1,2} & \alpha_{1,3} & \alpha_{1,4} \\
 \alpha_{2,1} & \alpha_{2,2} & \alpha_{2,3} & \alpha_{2,4} \\
 \alpha_{3,1} & \alpha_{3,2} & \alpha_{3,3} & \alpha_{3,4} \\
 \alpha_{4,1} & \alpha_{4,2} & \alpha_{4,3} & \alpha_{4,4}
\end{array}
\right)
\end{equation}
partial transposition with respect to second subsystems is obtained using
\code{PartialTransposeB[X, 2, 2]} and it gives
\begin{equation}
X^{T_B}=\left(
\begin{array}{cccc}
 \alpha_{1,1} & \alpha_{2,1} & \alpha_{1,3} & \alpha_{2,3} \\
 \alpha_{1,2} & \alpha_{2,2} & \alpha_{1,4} & \alpha_{2,4} \\
 \alpha_{3,1} & \alpha_{4,1} & \alpha_{3,3} & \alpha_{4,3} \\
 \alpha_{3,2} & \alpha_{4,2} & \alpha_{3,4} & \alpha_{4,4}
\end{array}
\right).
\end{equation}

\todo{dodac druga wersje}

%%%%%%%%%%%%%%%%%%%%%%%%%%%%%%%%%%%%%%%%%%%%%%%%%%%%%%%%%%%%%%%%%%%%%%%%%%%%%%%%
\subsubsection{Partial trace}
%%%%%%%%%%%%%%%%%%%%%%%%%%%%%%%%%%%%%%%%%%%%%%%%%%%%%%%%%%%%%%%%%%%%%%%%%%%%%%%%
Operation of tracing out a subsystem can be achieved in a similar manner. We 
define tracing map as
\begin{equation}
\Phi_\mathrm{tr}(\rho) = \tr \rho \Id,
\end{equation}
which for gives
\begin{equation}
D_{\Phi_\mathrm{tr}} =
\left(
\begin{array}{cccc}
 1 & 0 & 0 & 1 \\
 0 & 0 & 0 & 0 \\
 0 & 0 & 0 & 0 \\
 1 & 0 & 0 & 1
\end{array}
\right)
\end{equation}
for one qubit. Note that this matrix after reshuffling is equal to $\Id$ and 
thus $\Phi_\tr$ is CP-TP map.

\todo{dodac druga wersje}

%%%%%%%%%%%%%%%%%%%%%%%%%%%%%%%%%%%%%%%%%%%%%%%%%%%%%%%%%%%%%%%%%%%%%%%%%%%%%%%%
\subsubsection{Partial \SWAP}
%%%%%%%%%%%%%%%%%%%%%%%%%%%%%%%%%%%%%%%%%%%%%%%%%%%%%%%%%%%%%%%%%%%%%%%%%%%%%%%%
Partial transposition and partial trace are the most popular partial operations.

\todo{dodac przyklad}

%%%%%%%%%%%%%%%%%%%%%%%%%%%%%%%%%%%%%%%%%%%%%%%%%%%%%%%%%%%%%%%%%%%%%%%%%%%%%%%%
\section{Examples}\label{sec:examples}
%%%%%%%%%%%%%%%%%%%%%%%%%%%%%%%%%%%%%%%%%%%%%%%%%%%%%%%%%%%%%%%%%%%%%%%%%%%%%%%%

\ldots

%%%%%%%%%%%%%%%%%%%%%%%%%%%%%%%%%%%%%%%%%%%%%%%%%%%%%%%%%%%%%%%%%%%%%%%%%%%%%%%%
\subsection{Reshuffling}
%%%%%%%%%%%%%%%%%%%%%%%%%%%%%%%%%%%%%%%%%%%%%%%%%%%%%%%%%%%%%%%%%%%%%%%%%%%%%%%%

\ldots

%%%%%%%%%%%%%%%%%%%%%%%%%%%%%%%%%%%%%%%%%%%%%%%%%%%%%%%%%%%%%%%%%%%%%%%%%%%%%%%%
\subsection{Operator Schmidt decomposition}
%%%%%%%%%%%%%%%%%%%%%%%%%%%%%%%%%%%%%%%%%%%%%%%%%%%%%%%%%%%%%%%%%%%%%%%%%%%%%%%%

\ldots


%%%%%%%%%%%%%%%%%%%%%%%%%%%%%%%%%%%%%%%%%%%%%%%%%%%%%%%%%%%%%%%%%%%%%%%%%%%%%%%%
\section{Concluding remarks}\label{sec:comclude}
%%%%%%%%%%%%%%%%%%%%%%%%%%%%%%%%%%%%%%%%%%%%%%%%%%%%%%%%%%%%%%%%%%%%%%%%%%%%%%%%
Package \qi\ provide set of functions which aims to simplify the task of
exploring the space of quantum states and understanding quantum operations.

\ldots

%%%%%%%%%%%%%%%%%%%%%%%%%%%%%%%%%%%%%%%%%%%%%%%%%%%%%%%%%%%%%%%%%%%%%%%%%%%%%%%%
\section*{Acknowledgements}
%%%%%%%%%%%%%%%%%%%%%%%%%%%%%%%%%%%%%%%%%%%%%%%%%%%%%%%%%%%%%%%%%%%%%%%%%%%%%%%%
We acknowledge the financial support by Polish Research Network LFPPI.

\appendix
%%%%%%%%%%%%%%%%%%%%%%%%%%%%%%%%%%%%%%%%%%%%%%%%%%%%%%%%%%%%%%%%%%%%%%%%%%%%%%%%
\section{List of provided functions}
%%%%%%%%%%%%%%%%%%%%%%%%%%%%%%%%%%%%%%%%%%%%%%%%%%%%%%%%%%%%%%%%%%%%%%%%%%%%%%%%

\input{qi_functions}

\bibliography{qi}
\bibliographystyle{elsarticle-num}

\end{document}
\endinput
%%
%% End of file `elsarticle-template-num.tex'.
