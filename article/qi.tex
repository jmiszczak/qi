% v. 0.2 - 11/08/2009 - Jarek - extended introductions and bibliography
% v. 0.21 - 12/08/2009 - Jarek - preliminary sections
% v. 0.22 - 13/08/2009 - Jarek - added in intro and sec. 2
% v. 0.3 - 17/11/2009 - Jarek - added an appendix with a list of functions
% v. 0.31 - 18/11/2009 - Jarek - sections on states and operations
% v. 0.32 - 19/11/2009 - Jarek - improved abstact, applied CPC template 
% v. 0.33 - 24/09/2010 - Jarek - update basic info, sync with 0.3.18 version 
\documentclass[a4paper,11pt]{elsart}

\usepackage{hyperref}

%% This list environment is used for the references in the
%% Program Summary
%%
%\newcounter{bla}
%\newenvironment{refnummer}{%
%\list{[\arabic{bla}]}%
%{\usecounter{bla}%
% \setlength{\itemindent}{0pt}%
% \setlength{\topsep}{0pt}%
% \setlength{\itemsep}{0pt}%
% \setlength{\labelsep}{2pt}%
% \setlength{\listparindent}{0pt}%
% \settowidth{\labelwidth}{[9]}%
% \setlength{\leftmargin}{\labelwidth}%
% \addtolength{\leftmargin}{\labelsep}%
% \setlength{\rightmargin}{0pt}}}
% {\endlist}

%%%%%%%%%%%%%%%%%%%%%%%%%%%%%%%%%%%%%%%%%%%%%%%%%%%%%%%%%%%%%%%%%%%%%%%%%%%%%%%%
\usepackage{amsmath,amssymb}
\usepackage{dsfont}
\newcommand{\ket}[1]{\ensuremath{|#1\rangle}}
\newcommand{\bra}[1]{\ensuremath{\langle#1|}}
\newcommand{\Mathematica}{\emph{Mathematica}}
\newcommand{\wek}{\mathbf{vec}}
\newcommand{\res}{\mathbf{res}}
\newcommand{\1}{{\rm 1\hspace{-0.9mm}l}}
\newcommand{\Id}{\1}
\newcommand{\SWAP}{\ensuremath{\mathrm{SWAP}}}
\newcommand{\tr}{\mathrm{tr}}
\newcommand{\M}{\ensuremath{\mathbb{M}}}
\newcommand{\qi}{QI}
\newcommand{\code}[1]{\texttt{\small #1}}
\newcommand{\HS}[1]{\ensuremath{\mathcal{#1}}} % Hilbert space
\newcommand{\Cplx}{\ensuremath{\mathbb{C}}}
\newcommand{\eg}{\emph{eg.}}
\newcommand{\ie}{\emph{ie.}}
%\newcommand{\etal}{\emph{et al.}}
\newcommand{\Prob}[1]{\ensuremath{\mathrm{P}(#1)}}
\newcommand{\Observ}[1]{\ensuremath{#1}}
\newcommand{\Spectrum}[1]{\ensuremath{\sigma(#1)}}
\newcommand{\Spec}[1]{\Spectrum{#1}}
\newcommand{\ketbra}[2]{\ensuremath{\ket{#1}\bra{#2}}}
\newcommand{\proj}[1]{\ensuremath{\ketbra{#1}{#1}}}
\newcommand{\Proj}[1]{\proj{#1}}
\newcommand{\iner}[2]{\braket{#1}{#2}}
\newcommand{\Iner}[2]{\iner{#1}{#2}}
\newcommand{\braket}[2]{\ensuremath{\langle#1|#2\rangle}}
\newcommand{\Tr}[2][]{\ensuremath{\tr_{#1}{#2}}}
\newcommand{\id}{\mathds{1}}
\newcommand{\Space}[1]{\mathcal{#1}}
\newcommand{\SetOfStates}[1]{\ensuremath{\mathcal{S}(#1)}}
\newcommand{\States}[1]{\SetOfStates{#1}}
\newcommand{\Real}{\ensuremath{\mathds{R}}}
\newcommand{\N}{\ensuremath{\mathds{N}}}
\newcommand{\Z}{\ensuremath{\mathds{Z}}}
\newcommand{\set}[2]{\ensuremath{\left\{#1|#2\right\}}}
\newcommand{\re}[1]{\ensuremath{\mathrm{Re}\left(#1\right)}}
\newcommand{\im}[1]{\ensuremath{\mathrm{Im}\left(#1\right)}}
\newcommand{\Group}[2]{\ensuremath{(#1,#2)}}
\newcommand{\ord}[2]{\mathrm{ord}_#2(#1)}
\newcommand{\Var}[1]{\ensuremath{\mathrm{Var}(#1)}}
\newcommand{\halmos}{\newline\vspace{3mm}\hfill $\Box$}
\newcommand{\proof}{\noindent {\it Proof.\ }}
\newtheorem{theorem}{Theorem}

\usepackage{xcolor}
\newcommand{\todo}[1]{\textcolor{red}{\bf TODO: #1}}
%%%%%%%%%%%%%%%%%%%%%%%%%%%%%%%%%%%%%%%%%%%%%%%%%%%%%%%%%%%%%%%%%%%%%%%%%%%%%%%%

\journal{Computer Physics Communications}

\begin{document}

\begin{frontmatter}

%% Title, authors and addresses

%% use the tnoteref command within \title for footnotes;
%% use the tnotetext command for theassociated footnote;
%% use the fnref command within \author or \address for footnotes;
%% use the fntext command for theassociated footnote;
%% use the corref command within \author for corresponding author footnotes;
%% use the cortext command for theassociated footnote;
%% use the ead command for the email address,
%% and the form \ead[url] for the home page:
%% \title{Title\tnoteref{label1}}
%% \tnotetext[label1]{}
%% \author{Name\corref{cor1}\fnref{label2}}
%% \ead{email address}
%% \ead[url]{home page}
%% \fntext[label2]{}
%% \cortext[cor1]{}
%% \address{Address\fnref{label3}}
%% \fntext[label3]{}

\title{\qi: A package for the analysis of quantum states and operations in
\Mathematica}

\date{15/09/2011 (v. 0.34)}

\author{J.~A.~Miszczak\thanksref{author}},
\ead{miszczak@iitis.pl}
\author{P.~Gawron},
\ead{gawron@iitis.pl}
\author{Z.~Pucha{\l}a}
\ead{z.puchala@iitis.pl}

\thanks[author]{Corresponding author}

\address{Institute of Theoretical and Applied Informatics, Polish Academy of 
Sciences, Ba{\l}tycka 5, 44-100 Gliwice, Poland}

\begin{abstract}
QI is a package of functions for the \Mathematica\ computer algebra system,
which provides a framework for the analysis of quantum states and quantum
operations. In contrast to many available packages for symbolic and numerical
simulation of quantum computation presented package is focused on geometrical
aspects of quantum information theory. In particular \qi provides
parametrization of quantum states and selected families of quantum operation,
function for constructing composite quantum operations and methods for the
generation and analysis of random quantum operations. Also basic structures are
provided including the construction of ket vectors, basic unitary gates, random
states and unitaries, distance measures between quantum states and the selected
methods for the analysis of separability in quantum systems.

\begin{flushleft}
  %Insert your suggested PACS number here
%quantum information, symbolic computation (computer algebra) 
PACS: 03.67.-a; 02.70.Wz.
\end{flushleft}

\begin{keyword}
  % Please give some freely chosen keywords that we can use in a
  % cumulative keyword index.
quantum states; quantum channels; partial operations.
\end{keyword}
\end{abstract}

\end{frontmatter}
%%%%%%%%%%%%%%%%%%%%%%%%%%%%%%%%%%%%%%%%%%%%%%%%%%%%%%%%%%%%%%%%%%%%%%%%%%%%%%%%
{\bf PROGRAM SUMMARY}
  %Delete as appropriate.

\begin{small}
\noindent
{\em Manuscript Title:} \qi: A package for the analysis of quantum states and operations in
\Mathematica\\
{\em Authors:} J.A.~Miszczak, P.~Gawron, Z.~Pucha{\l}a \\
{\em Program Title:} QI \\
{\em Journal Reference:}                                      \\
  %Leave blank, supplied by Elsevier.
{\em Catalogue identifier:}                                   \\
  %Leave blank, supplied by Elsevier.
{\em Licensing provisions:} GPLv3 \\
{\em Programming language:} Mathematica 7\\
{\em Computer:} Any computer supporting Mathematica 7\\
  %Computer(s) for which program has been designed.
{\em Operating system:} Any operating system capable of running Mathematica 7 or higher, \eg\ GNU/Linux, MacOS X, FreeBSD, Microsoft Windows XP\\
  %Operating system(s) for which program has been designed.
{\em RAM:} bytes                                              \\
  %RAM in bytes required to execute program with typical data.
{\em Number of processors used:}                              \\
  %If more than one processor.
{\em Supplementary material:}                                 \\
  % Fill in if necessary, otherwise leave out.
{\em Keywords:} quantum states, quantum operations, partial operations  \\
  % Please give some freely chosen keywords that we can use in a
  % cumulative keyword index.
{\em PACS:} 03.67.-a, 02.70.Wz.\\
  % see http://www.aip.org/pacs/pacs.html
{\em Classification:} 4.15 \\
  %Classify using CPC Program Library Subject Index, see (
  % http://cpc.cs.qub.ac.uk/subjectIndex/SUBJECT_index.html)
  %e.g. 4.4 Feynman diagrams, 5 Computer Algebra.
{\em External routines/libraries:}                                      \\
  % Fill in if necessary, otherwise leave out.
{\em Subprograms used:}                                       \\
  %Fill in if necessary, otherwise leave out.

{\em Nature of problem:}\\
  %Describe the nature of the problem here.
  Construction of composed quantum operations, analysis of quantum states and
  operations.
   \\
{\em Solution method:}\\
  %Describe the method solution here.
  A package of functions for \Mathematica\ computer algebra system.
   \\
{\em Restrictions:}\\
  %Describe any restrictions on the complexity of the problem here.
  Running time of the presented procedures grows rapidliy with the dimensionalty
  of the problem.
   \\
%{\em Unusual features:}\\
%  %Describe any unusual features of the program/problem here.
%   \\
%{\em Additional comments:}\\
%  %Provide any additional comments here.
%   \\
{\em Running time:}\\
  %Give an indication of the typical running time here.
   \\

\end{small}

%\newpage

% In program descriptions the main text of the paper is listed under
% the heading LONG WRITE-UP.

%\hspace{1pc}
%{\bf LONG WRITE-UP}
%%%%%%%%%%%%%%%%%%%%%%%%%%%%%%%%%%%%%%%%%%%%%%%%%%%%%%%%%%%%%%%%%%%%%%%%%%%%%%%%


%\tableofcontents

%% \linenumbers

%% main text
%%%%%%%%%%%%%%%%%%%%%%%%%%%%%%%%%%%%%%%%%%%%%%%%%%%%%%%%%%%%%%%%%%%%%%%%%%%%%%%%
\section{Introduction}\label{sec:intro}
%%%%%%%%%%%%%%%%%%%%%%%%%%%%%%%%%%%%%%%%%%%%%%%%%%%%%%%%%%%%%%%%%%%%%%%%%%%%%%%%
Quantum information theory aims to harness the behavior of quantum mechanical
objects to store, transfer and process information~\cite{hayashi}. This behavior
is in many cases very different from the one we observe in classical world.
Quantum algorithms and protocols take advantage of superposition of states and
require the presence of entangled states. Both phenomena arise from the rich
structure of the space of quantum states. Hence, to explore the capabilities of
quantum information processing, one needs to fully understand this
space~\cite{BZ06}. 

Quantum mechanics provides us also with much larger than in classical case space
of allowed operations which can be used to manipulate quantum
states~\cite{hayashi,BZ06}. Recent results concerning additivity
problems~\cite{hastings09superadditivity} show that we are far from full
understanding the nature of quantum channels. Exploring the space of quantum
operations is fascinating, but cumbersome task.

We present a~package of functions developed for \Mathematica\ computing system
which aims to simplify the analysis of quantum states and quantum operations.
The package was developed in simplicity in mind and thus it uses only basic data
structures available in \Mathematica. This allows to relatively easily port
implemented functions to other scientific software systems. Also, in contrast to
most quantum computing packages
available~\cite{qdensity,qucalc,quantum2,qcwave}, \qi\ is not aimed to provide
tool for simulating quantum algorithms and protocols. We rather focus on the
analysis of quantum states used in those protocols and quantum channels, which
are used to describe allowed physical operations. The main goal of presented
package is to provide basic mathematical tools useful for studding geometrical
properties of quantum stats and quantum channels.

%This paper is organized as follows. In Section~\ref{sec:qi-intro} we review
%basic notion of quantum density matrices and quantum channels representing
%allowed physical transformations of density matrices. Section~\ref{sec:over}
%presents an overview of functionality provided by \qi\ package and describes
%some operations used as building blocks in specialized functions.
%Sections~\ref{sec:states} and \ref{sec:channels} provide detailed description of
%functions implemented in~\qi. Finally Section~\ref{sec:comclude} provides some
%concluding remarks.

%%%%%%%%%%%%%%%%%%%%%%%%%%%%%%%%%%%%%%%%%%%%%%%%%%%%%%%%%%%%%%%%%%%%%%%%%%%%%%%%
\section{Overview of \qi}\label{sec:over}
%%%%%%%%%%%%%%%%%%%%%%%%%%%%%%%%%%%%%%%%%%%%%%%%%%%%%%%%%%%%%%%%%%%%%%%%%%%%%%%%
\todo{troche o zalozeniach projektowych czyli dlaczego \qi\ jest inne niz
pozostale pakiety}

%%%%%%%%%%%%%%%%%%%%%%%%%%%%%%%%%%%%%%%%%%%%%%%%%%%%%%%%%%%%%%%%%%%%%%%%%%%%%%%%
\subsection{Design principles}
%%%%%%%%%%%%%%%%%%%%%%%%%%%%%%%%%%%%%%%%%%%%%%%%%%%%%%%%%%%%%%%%%%%%%%%%%%%%%%%%

\newcounter{principle}
\begin{list}{\textbf{P\arabic{principle}}}{\usecounter{principle}}
\item Functions are as simple as possible -- functions have minimum reasonable
number of arguments and implement small pieces of functionality. For example in
order to get partial trace on few subsystems user needs to perform permutation
and then apply partial trace 
\item User knows what she is doing -- functions implemented in \qi\ do not
validate the input and user is expected to provide reasonable data.
\item Only basic \Mathematica\ structures are used -- functions operator on
plain \Mathematica\ lists and return lists.
\item Some data are used more often then other -- \qi\ predefines some commonly
used matrices, for example matrices for \SWAP\ operation and Pauli matrices for
small dimensions.
\end{list}

%%%%%%%%%%%%%%%%%%%%%%%%%%%%%%%%%%%%%%%%%%%%%%%%%%%%%%%%%%%%%%%%%%%%%%%%%%%%%%%%
\section{Concluding remarks}\label{sec:comclude}
%%%%%%%%%%%%%%%%%%%%%%%%%%%%%%%%%%%%%%%%%%%%%%%%%%%%%%%%%%%%%%%%%%%%%%%%%%%%%%%%
Package \qi\ provide set of functions which aims to simplify the task of
exploring the space of quantum states and understanding quantum operations.

%%%%%%%%%%%%%%%%%%%%%%%%%%%%%%%%%%%%%%%%%%%%%%%%%%%%%%%%%%%%%%%%%%%%%%%%%%%%%%%%
\section*{Acknowledgements}
%%%%%%%%%%%%%%%%%%%%%%%%%%%%%%%%%%%%%%%%%%%%%%%%%%%%%%%%%%%%%%%%%%%%%%%%%%%%%%%%
We acknowledge the financial support by Polish Research Network LFPPI.

\appendix
%%%%%%%%%%%%%%%%%%%%%%%%%%%%%%%%%%%%%%%%%%%%%%%%%%%%%%%%%%%%%%%%%%%%%%%%%%%%%%%%
\section{List of provided functions}
%%%%%%%%%%%%%%%%%%%%%%%%%%%%%%%%%%%%%%%%%%%%%%%%%%%%%%%%%%%%%%%%%%%%%%%%%%%%%%%%

%\input{qi_functions}

\bibliography{qi}
\bibliographystyle{elsarticle-num}

\end{document}
\endinput
%%
%% End of file `elsarticle-template-num.tex'.
